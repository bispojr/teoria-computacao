\documentclass[xcolor=dvipsnames,table]{beamer}

\usepackage{latexsym}
\usepackage[utf8]{inputenc}
\usepackage[brazil]{babel}
\usepackage{amssymb}
\usepackage{amsmath}
\usepackage{stmaryrd}
\usepackage{fancybox}
\usepackage{datetime}
\usepackage[T1]{fontenc}
\usepackage{graphicx}
\usepackage{graphics}
\usepackage{url}
\usepackage{algorithmic}
\usepackage{algorithm}
\usepackage{acronym}
\usepackage{array}
\usepackage[normalem]{ulem}

\newtheorem{definicao}{Definio}
\newcommand{\tab}{\hspace*{2em}}

\mode<presentation>
{
  \definecolor{colortexto}{RGB}{0,0,0}
 
  \setbeamertemplate{background canvas}[vertical shading][ bottom=white!10,top=white!10]
  \setbeamercolor{normal text}{fg=colortexto} 

  \usetheme{Warsaw}
}

\title{Máquina de Turing} 

\author{
  Esdras Lins Bispo Jr. \\ \url{bispojr@ufg.br}
  } 
 \institute{
  Teoria da Computação \\Bacharelado em Ciência da Computação}
\date{\textbf{10 de maio de 2016} }

\logo{\includegraphics[width=1cm]{images/ufgJataiLogo.png}}

\begin{document}

	\begin{frame}
		\titlepage
	\end{frame}

	\AtBeginSection{
		\begin{frame}{Sumário}%[allowframebreaks]{Sumário}
    		\tableofcontents[currentsection]
    		%\tableofcontents[currentsection, hideothersubsections]
		\end{frame}
	}

	\begin{frame}{Plano de Aula}
		\tableofcontents
		%\tableofcontents[hideallsubsections]
	\end{frame}
    
    \begin{frame}[shrink]{Informações Importantes}
		\begin{block}{Testes}
			\begin{itemize}
				\item Teste 1 $\Rightarrow$ 20\% da pontuação total \\\sout{(16 de maio)} (17 de maio);
				\item Teste 2 $\Rightarrow$ 20\% da pontuação total \\\sout{(13 de junho)} (14 de junho);
				\item Teste 3 $\Rightarrow$ 20\% da pontuação total (28 de junho);
				\item Teste 4 $\Rightarrow$  20\% da pontuação total \\\sout{(08 de agosto)} (02 de agosto).
			\end{itemize}
		\end{block}
		\begin{block}{Avaliação}
			\begin{itemize}
				\item Prova $\Rightarrow$  20\% da pontuação total \\\sout{(16 e 30 de agosto)} (15 e 29 de agosto).
			\end{itemize}
		\end{block}
		\begin{block}{Exercícios [Bônus]}
			\begin{itemize}
				\item Somatório dos exercícios.
			\end{itemize}
		\end{block}
	\end{frame}
    
	\begin{frame}{Sorteio do Bônus (0,5 pt)}
		\begin{block}{Desafio}
			\begin{itemize}
				\item {\bf Problema 3.9 (a):} \\Mostre que 2-APs são mais poderosos que 1-APs. 
                \item Candidaturas até amanhã (10 de maio, 09h30); 
                \item Apresentação e resposta por escrito $\rightarrow$ \\segunda (16 de maio, 11h30); 
                \item 20 minutos de apresentação.
			\end{itemize}
		\end{block} 
        \begin{block}{Candidato escolhido}
        	???
        \end{block}
	\end{frame}
    
    \section{Pensamento}
	\begin{frame}{Pensamento}
  		\begin{center}
    		\includegraphics[width=7cm]{images/pensamento.png}
  		\end{center}
	\end{frame}
	
	\begin{frame}{Pensamento}
		\begin{columns}
			\column{.4\textwidth}  		
		  		\begin{center}
		    		\includegraphics[height=.5\textheight]{images/desconhecido.jpg}
		  		\end{center}
			\column{.6\textwidth}  		
				\begin{block}{Frase}
					\begin{center}
						{\large Ter a meta como alvo, mas viver pelo bom senso.}
					\end{center}
				\end{block}		  		
		  		\begin{block}{Quem?}
		  			\begin{center}
						{\bf Desconhecido} \\ ***
					\end{center}
				\end{block}
		\end{columns}
	\end{frame}
	
%------------------------------------------
	\section{Revisão}
    
    \begin{frame}{Definições}
		\begin{block}{Definição}
			Chame uma linguagem de {\bf Turing-reconhecível}, se alguma máquina de Turing a reconhece.
		\end{block} 
		\begin{block}{Definição}
			Chame uma linguagem de {\bf Turing-decidível}, se alguma máquina de Turing a decide.
		\end{block}
		\begin{block}{Corolário}
			Toda linguagem Turing-decidível é Turing-reconhecível.
		\end{block}
	\end{frame}		
	
	\begin{frame}{Exemplos}
		Uma máquina de Turing $M_2$ que decide $A = \{ 0^{2^n} \mbox{ | } n \geq 0 \}$: 
	\begin{center}
			\includegraphics[height=3.5cm]{images/m2.png}
		\end{center}
	\end{frame}
	
	\begin{frame}{Exemplos}
		\begin{block}{Descrição Formal de $M_2$}
			$M_2 = (Q, \Sigma, \Gamma, \delta, q_1 q_{aceita}, q_{rejeita})$:
			\begin{itemize}
				\item $Q = \{ q_1, q_2, q_3, q_4, q_5, q_{aceita}, q_{rejeita} \}$;
				\item $\Sigma = \{ 0 \}$,
				\item $\Gamma = \{ 0, x, \sqcup \}$,
				\item Descrevemos $\delta$ no próximo slide; e
				\item $q_1, q_{aceita}$ e $q_{rejeita}$ são o estado inicial, de aceitação e de rejeição, respectivamente.
			\end{itemize}
		\end{block}
	\end{frame}
	
	\begin{frame}{Exemplos}
	\begin{center}
			\includegraphics[height=7cm]{images/fig38.png}
		\end{center}
	\end{frame}
    
    \begin{frame}{Exemplos}
		\begin{block}{$L(M_1)$}
		Uma máquina de Turing $M_1$ que decide $B = \{ \omega \# \omega \mbox{ | } \omega \in \{ 0, 1 \}^* \}$		
		\end{block} 
		\begin{block}{Descrição Formal de $M_1$}
			$M_3 = (Q, \Sigma, \Gamma, \delta, q_1 q_{aceita}, q_{rejeita})$:
			\begin{itemize}
				\item $Q = \{ q_1, \ldots, q_{14}, q_{aceita}, q_{rejeita} \}$;
				\item $\Sigma = \{ 0, 1, \# \}$,
				\item $\Gamma = \{ 0, 1, \#, x, \sqcup \}$,
				\item Descrevemos $\delta$ no próximo slide; e
				\item $q_1, q_{aceita}$ e $q_{rejeita}$ são o estado inicial, de aceitação e de rejeição, respectivamente.
			\end{itemize}
		\end{block}
	\end{frame}
	
	\begin{frame}{Exemplos}
	\begin{center}
			\includegraphics[height=7cm]{images/fig310.png}
		\end{center}
	\end{frame}	
    
    \section{Máquina de Turing}	
	\begin{frame}{Exemplos}
		\begin{block}{$L(M_3)$}
		Uma máquina de Turing $M_3$ que decide $C = \{ a^i b^j c^k \mbox{ | } i \times j = k \mbox{ e } i,j,k \geq 1 \}$		
		\end{block}
	\end{frame}
	
	\begin{frame}{Exemplos}
	\begin{center}
			\includegraphics[height=5.5cm]{images/m3.png}
		\end{center}
	\end{frame}
	
	\begin{frame}{Exemplos}
		\begin{block}{$L(M_4)$}
		Uma máquina de Turing $M_3$ que reconhece $E = \{ \# x_1 \# x_2 \# \ldots \# x_l \mbox{ | cada } x_i \in \{ 0,1 \}^* \mbox{ e } x_i \not= x_j \mbox{ para cada } i \not= j \}$		
		\end{block}
	\end{frame}
	
	\begin{frame}{Exemplos}
	\begin{center}
			\includegraphics[height=3.5cm]{images/m4-1.png}
		\end{center}
	\end{frame}
	
	\begin{frame}{Exemplos}
	\begin{center}
			\includegraphics[height=4cm]{images/m4-2.png}
		\end{center}
	\end{frame}
    
    \subsection{MT Multifita}
	\begin{frame}[shrink]{MT Multifita}
		\begin{block}{Definição}
			Uma {\bf máquina de Turing multifita} é como uma máquina de Turing comum com várias fitas:
			\begin{itemize} \pause
				\item cada fita tem sua própria cabeça de leitura e escrita; \pause
				\item a configuração inicial consiste da cadeia de entrada aparecer sobre a fita 1, e as outras iniciar em branco; \pause
				\item a função de transição permite ler, escrever e mover as cabeças em algumas ou em todas as fitas simultaneamente
				\begin{center}
					$\delta : Q \times \Gamma^k \rightarrow Q \times \Gamma^k \times \{E,D,P\}^k$
				\end{center}
				em que $k$ é o número de fitas.
			\end{itemize}
		\end{block}\pause
		\begin{block}{Exemplo}
			$\delta(q_i, a_1, \ldots, a_k) = (q_j, b_1, \ldots, b_k, P, D, \ldots, E)$
		\end{block}
	\end{frame}
	
	\begin{frame}{MT Multifita}
		\begin{block}{Teorema}
			Toda máquina de Turing multifita tem uma máquina de Turing de uma única fita que lhe é equivalente.
		\end{block}
	\end{frame}
	
	\begin{frame}{MT Multifita}
		\begin{center}
			\includegraphics[height=6cm]{images/fig314.png}
		\end{center}
	\end{frame}
	
	\begin{frame}{MT Multifita}
		\begin{center}
			\includegraphics[height=5cm]{images/teoMt-1.png}
		\end{center}
	\end{frame}
	
	\begin{frame}{MT Multifita}
		\begin{center}
			\includegraphics[height=3.3cm]{images/teoMt-2.png}
		\end{center}
	\end{frame}
	
	\begin{frame}{MT Multifita}
		\begin{block}{Teorema}
			Toda máquina de Turing multifita tem uma máquina de Turing de uma única fita que lhe é equivalente.
		\end{block}
		\begin{block}{Corolário}
			Uma linguagem é Turing-reconhecível se e somente se alguma máquina de Turing multifita a reconhece.
		\end{block}
	\end{frame}
	
	\begin{frame}{MT Multifita}
		\begin{center}
			\includegraphics[height=1.8cm]{images/provaCorolario.png}
		\end{center}
	\end{frame}
	
	\begin{frame}
		\titlepage
	\end{frame}
	
\end{document}