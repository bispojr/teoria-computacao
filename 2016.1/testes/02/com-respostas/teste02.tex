\documentclass[12pt,a4paper,oneside]{article}

\usepackage[utf8]{inputenc}
\usepackage[portuguese]{babel}
\usepackage[T1]{fontenc}
\usepackage{amsmath}
\usepackage{amsfonts}
\usepackage{amssymb}
\usepackage{graphicx}
\usepackage{xcolor}
% Definindo novas cores
\definecolor{verde}{rgb}{0.25,0.5,0.35}

\author{\\Universidade Federal de Goiás (UFG) - Regional Jataí Jataí\\Bacharelado em Ciência da Computação \\Teoria da Computação \\Esdras Lins Bispo Jr.}

\date{14 de junho de 2016}

\title{\sc \huge Segundo Teste}

\begin{document}

\maketitle

{\bf ORIENTAÇÕES PARA A RESOLUÇÃO}

\begin{itemize}
	\item A avaliação é individual, sem consulta;
	\item A pontuação máxima desta avaliação é 10,0 (dez) pontos, sendo uma das 06 (seis) componentes que formarão a média final da disciplina: quatro testes, uma prova e exercícios;
	\item A média final ($MF$) será calculada assim como se segue
	\begin{eqnarray}
		MF & = & MIN(10, S) \nonumber \\
		S & = & (\sum_{i=1}^{4} 0,2.T_i ) + 0,2.P  + 0,1.E \nonumber
	\end{eqnarray}
	em que 
	\begin{itemize}
		\item $S$ é o somatório da pontuação de todas as avaliações,
		\item $T_i$ é a pontuação obtida no teste $i$,
		\item $P$ é a pontuação obtida na prova, e
		\item $E$ é a pontuação total dos exercícios.
	\end{itemize}
	\item O conteúdo exigido desta avaliação compreende o seguinte ponto apresentado no Plano de Ensino da disciplina: (2) Modelos de Computação e (3) Problemas Decidíveis.
\end{itemize}

\begin{center}
	\fbox{\large Nome: \hspace{10cm}}
	\fbox{\large Assinatura: \hspace{9cm}}
\end{center}

\newpage

\begin{enumerate}
	
	\section*{Segundo Teste}
	
	\item (5,0 pt) Seja $A = \{ \langle M \rangle$ | em que $M$ é uma expressão regular que não gera cadeias contendo um número par de $0$s$\}$. Mostre que $A$ é decidível.
	
	\vspace{0.3cm}
	
	{\color{verde}
		R - Primeiro será criado um AFD $S$ de forma que $L(S)$ seja formada por todas as cadeias que contém um número par de 0s. Admita que $\Gamma = \Sigma \setminus \{ 0 \}$ e que zero é um número par. É possível construir $S$, pois $L(S) = (\Gamma^* 0 \Gamma^* 0 \Gamma^*)^*$ é regular (Definição 1.16). 
		
		Ora, é necessário que $L(M) \cap L(S) = \emptyset$, pois $M$ não gera cadeias contendo um par de $0$s. Sabe-se também que $L(M) \cap L(S)$ é regular, pois $L(M)$ e $L(S)$ são regulares (Teorema 1.54 e Definição 1.16) e a classe de linguagens regulares é fechada sob intersecção. Logo, é possível construir o AFD $T$ de forma que $L(T) = L(M) \cap L(S)$ (Definição 1.16). 
		
		Diante disto, será construído a seguir um decisor $M_A$ para $A$ :
		
		$M_A$ = ``Sobre a entrada $\langle M \rangle$, em que $M$ é uma expressão regular, faça:
			\begin{enumerate}
				\item Construa o AFD $T$ conforme descrito anteriormente;
				\item Construa a MT $U$ que decide $V_{AFD}$ (Teorema 4.4);
				\item Rode $U$ sobre $\langle T \rangle$;
				\begin{enumerate}
					\item Se $U$ aceita, {\it aceite};
					\item Caso contrário, {\it rejeite}.
				\end{enumerate}					
			\end{enumerate}
		
		A linguagem $A$ é decidível pois foi possível construir uma máquina de Turing que a decide (Definição 3.6) $\blacksquare$
		
	}
	
	\newpage
	
	\item (5,0 pt) Seja $A = \{\langle M \rangle$ | $M$ é um AFN e $L(M) = 01 \cup 10 \}$. Mostre que $A$ é decidível.
	
	\vspace{0.3cm}
	
	{\color{verde}
		R - Primeiro será criado um AFD $S$ de forma que $L(S) = 01 \cup 10$. É possível construir $S$, pois $L(S)$ é regular (Definição 1.16). Ora, é necessário que $L(M) =  L(S)$ para que $\langle M \rangle \in A$. 	Diante disto, será construído a seguir um decisor $M_A$ para $A$ :
		
		$M_A$ = ``Sobre a entrada $\langle M \rangle$, em que $M$ é um AFN, faça:
			\begin{enumerate}
				\item Construa o AFD $S$ conforme descrito anteriormente;
				\item Converta o AFN $M$ no AFD $N$ (Teorema 1.39);
				\item Construa a MT $T$ que decide $EQ_{AFD}$ (Teorema 4.5);
				\item Rode $T$ sobre $\langle S, N \rangle$;
				\begin{enumerate}
					\item Se $T$ aceita, {\it aceite};
					\item Caso contrário, {\it rejeite}.
				\end{enumerate}					
			\end{enumerate}
		
		A linguagem $A$ é decidível pois foi possível construir uma máquina de Turing que a decide (Definição 3.6) $\blacksquare$
		
	}
	
\end{enumerate}

\section*{Teoremas Auxiliares}

\begin{itemize}
	
	\item[] {\bf Definição 1.16:} Uma linguagem é chamada de uma linguagem regular se algum autômato finito a reconhece.
	\item[] {\bf Teorema 1.25:} A classe de linguagens regulares é fechada sob a operação de união.
	\item[] {\bf Teorema 1.26:} A classe de linguagens regulares é fechada sob a operação de concatenação.
	\item[] {\bf Teorema 1.39:} Todo autômato finito não-determinístico tem um autômato finito determinístico
	equivalente.
	\item[] {\bf Teorema 1.49:} A classe de linguagens regulares é fechada sob a operação estrela.
	\item[] {\bf Teorema 1.54:} Uma linguagem é regular se e somente se alguma expressão regular a descreve.
	\item[] {\bf Definição 3.5:} Chame uma linguagem de Turing-reconhecível se alguma máquina de Turing a reconhece.
	\item[] {\bf Definição 3.6:} Chame uma linguagem de Turing-decidível ou simplesmente decidível se alguma máquina de Turing a decide.
	\item[] {\bf Teorema 3.13:} Toda máquina de Turing multifita tem uma máquina de Turing que lhe é equivalente.
	\item[] {\bf Teorema 3.16:} Toda máquina de Turing não-determinística tem uma máquina de Turing determinística que lhe é equivalente.
	\item[] {\bf Teorema 3.21:} Uma linguagem é Turing-reconhecível se e somente se algum enumerador a enumera.
	\item[] {\bf Teorema 4.1:} $A_{AFD}$ é uma linguagem decidível.
	\item[] {\bf Teorema 4.2:} $A_{AFN}$ é uma linguagem decidível.
	\item[] {\bf Teorema 4.3:} $A_{EXR}$ é uma linguagem decidível.
	\item[] {\bf Teorema 4.4:} $V_{AFD}$ é uma linguagem decidível.
	\item[] {\bf Teorema 4.5:} $EQ_{AFD}$ é uma linguagem decidível.
	\item[] {\bf Teorema 4.9:} Toda linguagem livre-de-contexto é decidível.
	\item[] {\bf Teorema 4.11:} $A_{MT}$ é uma linguagem indecidível.
	\item[] {\bf Definição 4.14:} Um conjunto $A$ é contável se é finito ou tem o mesmo tamanho que $N$.
\end{itemize}

\end{document}