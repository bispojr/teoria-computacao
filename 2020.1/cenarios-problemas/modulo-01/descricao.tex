\documentclass[a4paper, 11pt]{article}

\usepackage[bottom=2cm,top=3cm,left=3cm,right=2cm]{geometry}
\usepackage{setspace}
\usepackage{hyperref}
\usepackage{color}
\usepackage{tabularx}

\usepackage{multicol}

\title{
	{\sc \Huge Cenário do Problema PBL \\Módulo 01}
}
	
\author{ 
	    Universidade Federal de Jataí (UFJ)\\
	    Instituto de Ciências Exatas e Tecnológicas (ICET)\\
	    Bacharelado em Ciências da Computação (BCC)\\
	    Teoria da Computação
	
}
\date{01 de maio de 2020}

\onehalfspacing

\begin{document}
	
	\maketitle
	
	%\begin{multicols*}{2}
	
	\section{Descrição do Cenário}
	
	É bastante comum, em muitos sistemas atualmente, uma etapa de autenticação para fins de acesso. Para ocorrer a autenticação, é necessário normalmente um nome de usuário e uma senha. Costuma-se exigir que esta senha seja segura, isto é, seja difícil de ser descoberta com um número pequeno de tentativas. Além disto, espera-se que o armazenamento desta senha no servidor também seja seguro, de forma que os dados armazenados estejam devidamente criptografados.
	
	Deseja-se mostrar o poder das máquinas de Turing (MT) pem relação a este cenário. Logo, além da construção desta MT, será necessário visualizar o seu funcionamento.
	
	\section{Artefatos Esperados}
	
	    \begin{itemize}
	        \item Documento de especificação dos requisitos (via {\it user stories}) desta MT;
	        \item Documento com a descrição formal (i.e. a 7-upla) desta MT;
	        \item Execução desta MT em um simulador visual;
	        \item Relatório da resolução do problema, elencando explicitamente as hipóteses adotadas.
	    \end{itemize}
	
	\section{Datas Importantes}
	
	\begin{itemize}
	    \item[] {\bf [24/03]} Especificação do Problema (a\-cordado com o professor da disciplina);
	    \item[] {\bf [13/04]} Entrega do Relatório até 12h00 (via Google Classroom);
	    \item[] {\bf [13/04]} Apresentação da Resolução de até 15 minutos (Sala de Aula).
	\end{itemize}
	
	\section{Recursos}
	
	Sugestão de alguns recursos interessantes.
	
	\subsection{Livros}
	
	    \begin{itemize}
	        \item SIPSER, M. {\bf Introdução à Teoria da Computação}, 2a Edição, Editora Thomson Learning, 2011. {\color{blue} \bf Código Bib.: [004 SIP/int]}.
	        \item HOPCROFT, J.E.; MOTWANI, R.; ULLMAN, J. D. {\bf Introdução à teoria de autô\-matos, linguagens e computação}. Campus Elsevier, 2002. {\color{blue} \bf Código Bib.: [004.43 HOP/int]}.
	        \item LEWIS, H. R; PAPADIMITRIOU, C. H. {\bf Elementos de Teoria da Computação}. 2. ed. - Porto Alegre (RS): Bookman., 2000. {\color{blue} \bf Código Bib.: [004 LEW /ele 2.ed.]}.
	    \end{itemize}
	
	\subsection{Simuladores}
	
	\begin{itemize}
	    \item \url{https://turingmachinesimulator.com/}
	    \item \url{https://turingmachine.io/}
	    \item \url{http://morphett.info/turing/turing.html}
	\end{itemize}
	
	\subsection{Documentação de Requisitos}
	
	\begin{itemize}
	    \item Histórias de Usuário: \url{https://www.youtube.com/watch?v=sEtiCJfXTE8};
	    \item {\it Template} para {\it User Stories}?: \url{https://www.youtube.com/watch?v=EgjHUKpVgRM}.
	\end{itemize}
	
	%\end{multicols*}
	
\end{document}