\documentclass[xcolor=dvipsnames,table]{beamer}

\usepackage{latexsym}
\usepackage[utf8]{inputenc}
\usepackage[brazil]{babel}
\usepackage{amssymb}
\usepackage{amsmath}
\usepackage{stmaryrd}
\usepackage{fancybox}
\usepackage{datetime}
\usepackage[T1]{fontenc}
\usepackage{graphicx}
\usepackage{graphics}
\usepackage{url}
\usepackage{algorithmic}
\usepackage{algorithm}
\usepackage{acronym}
\usepackage{array}

\newtheorem{definicao}{Definio}
\newcommand{\tab}{\hspace*{2em}}

\mode<presentation>
{
  \definecolor{colortexto}{RGB}{0,0,0}
 
  \setbeamertemplate{background canvas}[vertical shading][ bottom=white!10,top=white!10]
  \setbeamercolor{normal text}{fg=colortexto} 

  \usetheme{Warsaw}
}

\title{Decidibilidade} 

\author{
  Esdras Lins Bispo Jr. \\ \url{bispojr@ufg.br}
  } 
 \institute{
  Teoria da Computação \\Bacharelado em Ciência da Computação}
\date{\textbf{05 de junho de 2017} }

\logo{\includegraphics[width=1cm]{images/ufgJataiLogo.png}}

\begin{document}

	\begin{frame}
		\titlepage
	\end{frame}

	\AtBeginSection{
		\begin{frame}{Sumário}%[allowframebreaks]{Sumário}
    		\tableofcontents[currentsection]
    		%\tableofcontents[currentsection, hideothersubsections]
		\end{frame}
	}

	\begin{frame}{Plano de Aula}
		\tableofcontents
		%\tableofcontents[hideallsubsections]
	\end{frame}
	
	\begin{frame}{Bônus (0,5 pt)}
		\begin{block}{Desafio}
			\begin{itemize}
			\item {\bf Problema 4.19:} \\Seja $S = \{ \langle M \rangle \mbox{ | } M$ é um AFD que aceita $\omega^R$ sempre que ele aceita $\omega \}$. Mostre que $S$ é decidível.
				\item Candidaturas agora; 
				\item Apresentação e resposta por escrito $\rightarrow$ \\Quinta (08 de junho, 15h30); 
				\item 10 minutos de apresentação.
			\end{itemize}
		\end{block} 
		\begin{block}{Livro}
			SIPSER, M. {\bf Introdução à Teoria da Computação}, 2a Edição, Editora Thomson Learning, 2011. \color{blue}{\bf Código Bib.: [004 SIP/int]}.
		\end{block}
	\end{frame}
    
    \section{Revisão}
	
	\subsection{Terminologia para descrever MTs}
	\begin{frame}{Terminologia para descrever MTs}
		\begin{block}{Níveis de descrição}
			\begin{itemize}
				\item {\bf Descrição formal}: esmiúça todos os elementos da 7-upla, conforme definição;
				\item {\bf Descrição de implementação}: descreve a forma pela qual a MT move a sua cabeça e a forma como ela armazena os dados na fita;
				\item {\bf Descrição de alto nível}: neste nível não precisamos mencionar como a máquina administra a sua fita ou sua cabeça de leitura-escrita.
			\end{itemize}
		\end{block}
	\end{frame}
	
	\begin{frame}[shrink]{Exemplo}
		Seja $A$ a linguagem consistindo em todas as cadeias representando grafos não-direcionados que são conexos. Logo:
		\begin{center}
			$A = \{\langle G \rangle | G$ é um grafo não-direcionado conexo$\}$
		\end{center}
		\begin{block}{Descrição de alto nível}
			$M$ = ``Sobre a entrada $\langle G \rangle$, a codificacão de um grafo $G$:
			\begin{enumerate}
				\item Selecione o primeiro nó de $G$ e marque-o.
				\item Repita o seguinte estágio até que nenhum novo nó seja marcado:
				\begin{enumerate}
					\item Para cada nó em $G$, marque-o se ele está ligado por uma aresta a um nó que já está marcado.
				\end{enumerate}
				\item Faça uma varredura em todos os nós de $G$ para determinar se eles estão todos marcados. Se eles estão, {\it aceite}; caso contrário, {\it rejeite}''.
			\end{enumerate}
		\end{block}
	\end{frame}
	
	\begin{frame}{Exemplo}
		\begin{block}{Pergunta}
			Como seria a descrição de $M$ no nível de implementação?
		\end{block}
	\end{frame}

	\subsection{Decidibilidade}
	\begin{frame}{Introdução}
		\begin{block}{Propósitos da Teoria da Computação}
			\begin{itemize}
				\item Conhecer o poder dos algoritmos;
				\item Explorar os limites da solubilidade algorítmica;
				\item Identificar algoritmos insolúveis.
			\end{itemize}
		\end{block}
		\begin{block}{Por que devemos estudar insolubilidade?}
			\begin{itemize}
				\item Relaxamento dos requisitos;
				\item Conhecimento das limitações dos modelos computacionais.
			\end{itemize}
		\end{block}
	\end{frame}

	\section{Decidibilidade}
	
	\begin{frame}{Linguagens Decidíveis}
		\begin{block}{Exemplos de Linguagens Decidíveis}
			São úteis porque
			\begin{itemize}
				\item Algumas linguagens decidíveis estão associadas a aplicações;
				\item Algumas linguagens aparentemente triviais não são decidíveis.
			\end{itemize}
		\end{block} \pause
		\begin{block}{Problema da aceitação}
			Dado um modelo computacional $MC$ e uma cadeia de entrada $\omega$, identificar se $MC$ aceita $\omega$.
		\end{block}	
	\end{frame}
	
	\begin{frame}{Problema da aceitação para AFDs}
		\begin{block}{Problema da aceitação para AFDs}
			Dado um AFD $B$ e uma cadeia de entrada $\omega$, identificar se $B$ aceita $\omega$.
		\end{block}	\pause
		\begin{block}{Problema}
			$A_{AFD} = \{ \langle B, \omega \rangle \mbox{ | } B \mbox{ é um AFD que aceita a cadeia de entrada } \omega \}$
		\end{block} \pause
		\begin{block}{Estratégia de Resolução}
			Resolver o problema da aceitação para AFDs é decidir se $\omega \in A_{AFD}$.
		\end{block}
	\end{frame}
	
	\begin{frame}{Problema da aceitação para AFDs}
		\begin{block}{Teorema 4.1}
			$A_{AFD}$ é uma linguagem decidível.
		\end{block} \pause
		\begin{block}{Ideia da Prova}
			$M$ = ``Sobre a entrada $\langle B, \omega \rangle$, em que $B$ é um AFD, e $\omega$, uma cadeia:
			\begin{enumerate}
				\item Simule $B$ sobre a entrada $\omega$;
				\item Se a simulação termina em um estado de aceitação, {\bf aceite}. Senão, {\bf rejeite}.''
			\end{enumerate}
		\end{block}
	\end{frame}
	
	\begin{frame}{Problema da aceitação para AFDs}
		\begin{block}{Detalhes de implementação}
			\begin{itemize}
				\item A entrada $\langle B, \omega \rangle$ representa um AFD e uma cadeia;
				\begin{itemize}
					\item Uma representação razoável de $B$ seria uma lista de seus cinco componentes: $Q, \Sigma, \delta, q_0$ e $F$;
					\item $M$ simula $B$ de forma que $M$ {\bf aceita} se $B$ estiver em um estado final, e {\bf rejeita}, caso contrário.
				\end{itemize}
			\end{itemize}
		\end{block}
	\end{frame}
	
	\begin{frame}{Problema da aceitação para AFNs}
		\begin{block}{Problema da aceitação para AFNs}
			Dado um AFN $B$ e uma cadeia de entrada $\omega$, identificar se $B$ aceita $\omega$.
		\end{block}	\pause
		\begin{block}{Problema}
			$A_{AFN} = \{ \langle B, \omega \rangle \mbox{ | } B \mbox{ é um AFN que aceita a cadeia de entrada } \omega \}$
		\end{block} \pause
		\begin{block}{Estratégia de Resolução}
			Decidir se $\langle B, \omega \rangle \in A_{AFN}$.
		\end{block}
	\end{frame}	
	
	\begin{frame}{Problema da aceitação para AFNs}
		\begin{block}{Teorema 4.2}
			$A_{AFN}$ é uma linguagem decidível.
		\end{block} \pause
		\begin{block}{Prova}
			$N$ = ``Sobre a entrada $\langle B, \omega \rangle$, em que $B$ é um AFN, e $\omega$, uma cadeia:
			\begin{enumerate}
				\item Converta AFN $B$ para um AFD equivalente $C$, usando o procedimento para essa conversão dado no Teorema 1.39;
				\item Rode a MT $M$ do Teorema 4.1 sobre a cadeia $\langle C, \omega \rangle$;
				\item Se $M$ aceita, {\bf aceite}. Caso contrário, {\bf rejeite}.''
			\end{enumerate}
		\end{block}
	\end{frame}
	
	\begin{frame}{Problema da Vacuidade de uma Linguagem}
		\begin{block}{Descrição}
			Dada uma linguagem $L$, identificar se $L = \emptyset$.
		\end{block}	
		\begin{block}{Problema aplicado a AFDs}
			$V_{AFD} = \{ \langle A \rangle \mbox{ | } A \mbox{ é um AFD e } L(A) = \emptyset \}$
		\end{block} 
		\begin{block}{Estratégia de Resolução}
			Decidir se $\langle A \rangle \in V_{AFD}$.
		\end{block}
	\end{frame}	
	
	\begin{frame}{Problema da Vacuidade de uma Linguagem}
		\begin{block}{Teorema 4.4}
			$V_{AFD}$ é uma linguagem decidível.
		\end{block} \pause
		\begin{block}{Prova}
			A seguinte MT $T$ decide $V_{AFD}$.
			
			$T$ = ``Sobre a entrada $\langle A \rangle$, em que $A$ é uma AFD:
			\begin{enumerate}
				\item Marque o estado inicial de $A$;
				\item Repita até que nenhum estado novo venha a ser marcado;
				\begin{enumerate}
					\item Marque qualquer estado que tenha uma transição chegando nele a partir de qualquer estado que já está marcado.
				\end{enumerate}
				\item Se nenhum estado final estiver marcado, {\bf aceite}. Caso contrário, {\bf rejeite}.''
			\end{enumerate}
		\end{block}
	\end{frame}
	
	\begin{frame}
		\titlepage
	\end{frame}
	
\end{document}