\documentclass[12pt,a4paper,oneside]{article}

\usepackage[utf8]{inputenc}
\usepackage[portuguese]{babel}
\usepackage[T1]{fontenc}
\usepackage{amsmath}
\usepackage{amsfonts}
\usepackage{amssymb}

\usepackage{multirow}
\usepackage{array,graphicx}

\usepackage{xcolor}
% Definindo novas cores
\definecolor{verde}{rgb}{0.25,0.5,0.35}
\definecolor{jpurple}{rgb}{0.5,0,0.35}

\author{\\Universidade Federal de Goiás (UFG) - Regional Jataí\\Bacharelado em Ciência da Computação \\Teoria da Computação - 2017.2 \\Prof. Esdras Lins Bispo Jr.}
\date{}

\title{
	\sc \huge Lista de Exercícios (Excepcional)
	\\{\tt Versão 1.0}
}

\begin{document}

\maketitle

\section{Livro de Referência}
	\begin{itemize}
		\item SIPSER, M. {\bf Introdução à Teoria da Computação}, 2a Edição, Editora Thomson Learning, 2011. \color{blue}{\bf Código Bib.: [004 SIP/int]}.
	\end{itemize}
	
\section{Lista de Exercícios}

\begin{enumerate}
	\item[] {\bf Teste 01:} 3.9, 3.15;
	\item[] {\bf Teste 02:} 3.13, 3.16;
	\item[] {\bf Teste 03:} 4.12, 4.15;
	\item[] {\bf Teste 04:} 7.6, 7.19.
	
\end{enumerate}

\end{document}