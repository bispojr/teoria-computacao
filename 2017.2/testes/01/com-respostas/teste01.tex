\documentclass[12pt,a4paper,oneside]{article}

\usepackage[utf8]{inputenc}
\usepackage[portuguese]{babel}
\usepackage[T1]{fontenc}
\usepackage{amsmath}
\usepackage{amsfonts}
\usepackage{amssymb}
\usepackage{graphicx}
\usepackage{xcolor}
% Definindo novas cores
\definecolor{verde}{rgb}{0.25,0.5,0.35}

\author{\\Universidade Federal de Goiás (UFG) - Regional  Jataí\\Bacharelado em Ciência da Computação \\Teoria da Computação \\Esdras Lins Bispo Jr.}

\date{16 de novembro de 2017}

\title{\sc \huge Primeiro Teste}

\begin{document}

\maketitle

{\bf ORIENTAÇÕES PARA A RESOLUÇÃO}

\small
 
\begin{itemize}
	\item A avaliação é individual, sem consulta;
	\item A pontuação máxima desta avaliação é 10,0 (dez) pontos, sendo uma das 06 (seis) componentes que formarão a média final da disciplina: quatro testes, uma prova e exercícios;
	\item A média final ($MF$) será calculada assim como se segue
	\begin{eqnarray}
		MF & = & MIN(10, S) \nonumber \\
		S & = & (\sum_{i=1}^{4} 0,2.T_i ) + 0,2.P  + EB\nonumber
	\end{eqnarray}
	em que 
	\begin{itemize}
		\item $S$ é o somatório da pontuação de todas as avaliações,
		\item $T_i$ é a pontuação obtida no teste $i$,
		\item $P$ é a pontuação obtida na prova, e
		\item $EB$ é a pontuação total dos exercícios-bônus.
	\end{itemize}
	\item O conteúdo exigido desta avaliação compreende o seguinte ponto apresentado no Plano de Ensino da disciplina: (1) Teoria da Computação e (2) Modelos de Computação.
\end{itemize}

\begin{center}
	\fbox{\large Nome: \hspace{10cm}}
\end{center}

\newpage

\begin{enumerate}
	
	\section*{Primeiro Teste}
	
	\item (5,0 pt) {\bf [Sipser 3.9 (a)]} Seja um $k$-AP um autômato com pilha que tem $k$ pilhas. Portanto, um 0-AP é um AFN e um 1-AP é um AP convencional. Você já sabe que 1-APs são mais poderosos (reconhecem uma classe maior de linguagens) que 0-APs. Agora, mostre que 2-APs são mais poderosos que 1-APs.	
	
	\vspace{0.3cm}
	
	{\color{blue}	
		
		{\bf Prova:} Iremos realizar esta demonstração em dois passos:
		\begin{enumerate}
			\item 2-APs são pelo menos tão poderosos que 1-APs; e
			\item É possível construir um 2-AP que reconhece $A = \{0^n1^n2^n$ | $n \geq 0\}$
		\end{enumerate} 
	
		\underline{Prova do passo (a):}  Ora é possível simular qualquer 1-AP em um 2-AP. Para isto, basta reproduzir todo o mecanismo do 1-AP em um 2-AP desconsiderando uma das pilhas. Desta forma, 2-APs são pelo menos tão poderosos que 1-APs. $\blacksquare$
		
		\underline{Prova do passo (b):} O 2-AP $M$, descrito em alto nível abaixo, reconhece a linguagem $A$.
		
		$M$ = ``Sobre a entrada $\omega$, faça:
		\begin{enumerate}
			\item Se $\omega = \epsilon$, aceite.
			\item Se o primeiro símbolo for 0, leia e empilhe todos os 0s consecutivos na pilha 1. Caso contrário, rejeite (enviando a execução da máquina para um estado de fuga).
			\item Se o próximo símbolo for 1, leia e empilhe todos os 1s consecutivos na pilha 2. Caso contrário, rejeite.
			\item Se o próximo símbolo for 2, leia cada símbolo 2, desempilhando simultaneamente um símbolo da pilha 1 e um símbolo da pilha 2. Caso contrário, rejeite.
			\begin{enumerate}
				\item Se, neste passo, não for possível desempilhar uma das pilhas a cada leitura do símbolo 2, rejeite;
				\item Se, neste passo, encerrou a leitura de símbolos 2 e uma das pilhas tiver elementos, rejeite;
			\end{enumerate}
			\item Se ainda houver símbolos a serem lidos, rejeite;
			\item Aceite.		
		\end{enumerate} 
	
	Como a linguagem $A$ não é livre-de-contexto, nenhum 1-AP a reconhece. Como foi possível construir M (passo (b)), e sabendo do passo (a), podemos afirmar que 2-APs são mais poderosos que 1-APs. $\blacksquare$
	}

	\newpage
	
	\item (5,0 pt) A operação binária ou-exclusivo, representada pelo símbolo $\otimes$, é definida da seguinte forma:
	\begin{center}
		$X \otimes Y = (\overline{X} \cap Y) \cup (X \cap \overline{Y})$
	\end{center}
	em que $X$ e $Y$ são dois conjuntos quaisquer.
	
	Mostre que a classe de linguagens decidíveis é fechada sob a operação de ou-exclusivo.
	
	\vspace{0.3cm}
	
	{\color{blue}	
		{\bf Prova:} Sejam $A$ e $B$ duas linguagens decidíveis. É possível construir duas máquinas de Turing (MTs) $M_A$ e $M_B$ que decidem as linguagens $A$ e $B$, respectivamente (pois se uma linguagem é decidível, então uma  MT a decide). Iremos construir a MT $M_{aux}$, a partir de $M_A$ e $M_B$, que decide $A \otimes B$. A descrição de $M_{aux}$ é dada a seguir:
		
		$M_{aux}$ = ``Sobre a entrada $\omega$, faça:
		\begin{enumerate}
			\item Rode $M_A$ sobre $\omega$; 
			\item Rode $M_B$ sobre $\omega$; 
			\item Se $M_A$ rejeita e $M_B$ aceita, {\it aceite};
			\item Se $M_A$ aceita e $M_B$ rejeita, {\it aceite};
			\item {\it Rejeite}.''.
		\end{enumerate}
		
		Como foi possível construir $M_{aux}$, então $A \otimes B$ é decidível. Ora, se $A \otimes B$ é decidível, então a classe de linguagens decidíveis é fechada sob a operação de ou-exclusivo $\blacksquare$
	}

\end{enumerate}

\end{document}