\documentclass[12pt,a4paper,oneside]{article}

\usepackage[utf8]{inputenc}
\usepackage[portuguese]{babel}
\usepackage[T1]{fontenc}
\usepackage{amsmath}
\usepackage{amsfonts}
\usepackage{amssymb}
\usepackage{graphicx}
\usepackage{xcolor}
% Definindo novas cores
\definecolor{verde}{rgb}{0.25,0.5,0.35}

\author{\\Universidade Federal de Goiás (UFG) - Regional  Jataí\\Bacharelado em Ciência da Computação \\Teoria da Computação \\Esdras Lins Bispo Jr.}

\date{14 de dezembro de 2017}

\title{\sc \huge Segundo Teste}

\begin{document}

\maketitle

{\bf ORIENTAÇÕES PARA A RESOLUÇÃO}

\small
 
\begin{itemize}
	\item A avaliação é individual, sem consulta;
	\item A pontuação máxima desta avaliação é 10,0 (dez) pontos, sendo uma das 06 (seis) componentes que formarão a média final da disciplina: quatro testes, uma prova e exercícios;
	\item A média final ($MF$) será calculada assim como se segue
	\begin{eqnarray}
		MF & = & MIN(10, S) \nonumber \\
		S & = & (\sum_{i=1}^{4} 0,2.T_i ) + 0,2.P  + EB\nonumber
	\end{eqnarray}
	em que 
	\begin{itemize}
		\item $S$ é o somatório da pontuação de todas as avaliações,
		\item $T_i$ é a pontuação obtida no teste $i$,
		\item $P$ é a pontuação obtida na prova, e
		\item $EB$ é a pontuação total dos exercícios-bônus.
	\end{itemize}
	\item O conteúdo exigido desta avaliação compreende o seguinte ponto apresentado no Plano de Ensino da disciplina: (2) Modelos de Computação, e (3) Problemas Decidíveis.
\end{itemize}

\begin{center}
	\fbox{\large Nome: \hspace{10cm}}
\end{center}

\newpage

\begin{enumerate}
	
	\section*{Segundo Teste}
	
	\item (5,0 pt)  {\bf [Sipser 3.6]} No Teorema 3.21, mostramos que uma linguagem é Turing-reconhecível sse algum enumerador a enumera. Por que não usamos o seguinte algoritmo mais simples
	para a direção de ida da prova? Tal qual anteriormente, $s_1, s_2, \ldots$ é uma lista de todas as cadeias em $\Sigma^*$.
	
	$E$ = ``Ignore a entrada.
	
	\begin{enumerate}
		\item Repita o que se segue para $i = 1,2,3,...$
		\item Rode $M$ sobre $s_i$ .
		\item Se ela aceita, imprima $s_i$ .
	\end{enumerate}

	\item (5,0 pt) A partir da classe de linguagens Turing-reconhecíveis, mostre 
	\begin{enumerate}
		\item o fecho sob alguma operação; e
		\item a impossibilidade do fecho sob a operação de complemento.
	\end{enumerate}
	
	

\end{enumerate}

\end{document}