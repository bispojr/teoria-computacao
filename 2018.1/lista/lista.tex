\documentclass[12pt,a4paper,oneside]{article}

\usepackage[utf8]{inputenc}
\usepackage[portuguese]{babel}
\usepackage[T1]{fontenc}
\usepackage{amsmath}
\usepackage{amsfonts}
\usepackage{amssymb}

\usepackage{multirow}
\usepackage{array,graphicx}

\usepackage{xcolor}
% Definindo novas cores
\definecolor{verde}{rgb}{0.25,0.5,0.35}
\definecolor{jpurple}{rgb}{0.5,0,0.35}

\author{\\Universidade Federal de Jataí (UFJ)\\Bacharelado em Ciência da Computação \\Teoria da Computação - 2018.1 \\Prof. Esdras Lins Bispo Jr.}
\date{}

\title{
	\sc \huge Lista de Exercícios
	\\{\tt Versão 1.0}
}

\begin{document}

\maketitle

\section{Livro de Referência}
	\begin{itemize}
		\item SIPSER, M. {\bf Introdução à Teoria da Computação}, 2a Edição, Editora Thomson Learning, 2011. \color{blue}{\bf Código Bib.: [004 SIP/int]}.
	\end{itemize}
	
\section{Exercícios}

\begin{enumerate}
	\item[] {\bf Capítulo 3:} 3.1, 3.2, 3.4, 3.5, 3.6, 3.7, 3.8, 3.9, 3.15, 3.16;
	\item[] {\bf Capítulo 4:} 4.1, 4.2, 4.3, 4.9, 4.6, 4.7, 4.11, 4.12;
	\item[] {\bf Capítulo 7:} 7.1, 7.6, 7.7, 7.8, 7.9, 7.10, 7.11.
	
\section{Metodologia}
	\begin{itemize}
		\item Encontros de 30 minutos para a apresentação das respostas;
		\item Para cada questão, 40\% corresponde à resposta escrita e \\60\% à apresentação individual para o professor;
		\item O somatório da pontuação obtida por todas as questões compreende a pontuação total da lista de exercícios;
		\item É possível refazer o exercício a qualquer momento com o propósito de melhorar a pontuação referente àquele exercício.
	\end{itemize}
	

	
\end{enumerate}

\end{document}